\documentclass[a4paper,11pt]{article}

%A Few Useful Packages
\usepackage{marvosym}
\usepackage[math-style=TeX]{unicode-math}
\usepackage{fontspec} 					%for loading fonts
\usepackage{xunicode,xltxtra,url,parskip} 	%other packages for formatting
\usepackage{color,graphicx}
\usepackage[usenames,dvipsnames]{xcolor}
\usepackage[big]{layaureo} 				%better formatting of the A4 page
% an alternative to Layaureo can be ** \usepackage{fullpage} **
\usepackage{supertabular} 				%for Grades
\usepackage{titlesec}					%custom \section
\usepackage{ragged2e}

%Setup hyperref package, and colours for links
\usepackage{hyperref}
\definecolor{linkcolour}{rgb}{0,0.2,0.6}
\hypersetup{
  pdftitle={赵磊的简历},
  pdfauthor={赵磊},
  pdfkeywords={简历},
  pdfcreator={XeLaTeX},
  hidelinks
}

%FONTS
\defaultfontfeatures{Mapping=tex-text}
%\setmainfont[SmallCapsFont = Fontin SmallCaps]{Fontin}
%%% modified for Karol Kozioł for ShareLaTeX use
\setmainfont{Fontin.otf}[
  SmallCapsFont = Fontin-SmallCaps.otf,
  BoldFont = Fontin-Bold.otf,
  ItalicFont = Fontin-Italic.otf
]
\setmathfont{texgyrepagella-math.otf}
%%%

\usepackage{xeCJK}
\setCJKmainfont{Noto Sans CJK SC}

\usepackage{metalogo}
\setlogokern{La}{-0.18em}
\setlogokern{aT}{-0.09em}
\setlogodrop[TeX]{0.27ex}

\hyphenpenalty=500
\frenchspacing

%CV Sections inspired by:
%http://stefano.italians.nl/archives/26
\titleformat{\section}{\Large\scshape\raggedright}{}{0em}{}[\titlerule]
\titlespacing{\section}{0pt}{3pt}{3pt}
%Tweak a bit the top margin
%\addtolength{\voffset}{-1.3cm}

%Italian hyphenation for the word: ''corporations''
\hyphenation{im-pre-se}

%-------------WATERMARK TEST [**not part of a CV**]---------------
\usepackage[absolute]{textpos}

\setlength{\TPHorizModule}{30mm}
\setlength{\TPVertModule}{\TPHorizModule}
\textblockorigin{2mm}{0.65\paperheight}
\setlength{\parindent}{0pt}

%--------------------BEGIN DOCUMENT----------------------
\begin{document}


\pagestyle{empty} % non-numbered pages


%--------------------TITLE-------------
\par{\centering
		{\Huge 赵 磊
	}\bigskip\par}

%--------------------SECTIONS-----------------------------------
%Section: Personal Data
\section{个人信息}

\begin{tabular}{rl}
  % \textsc{Place and Date of Birth:} & Someplace, Italy  | dd Month 1912 \\
  \textsc{地址}          & 上海市浦东新区 \\
  \textsc{手机}          & +1 319 555 0128 \\
  \textsc{电邮}          & \href{mailto:lei.zhao@example.com}{\color{linkcolour}lei.zhao@example.com} \\
  \textsc{StackOverflow} & \href{https://stackoverflow.com/users/2117531/lei-zhao}{\color{linkcolour}stackoverflow.com/u/2117531} \\
  \textsc{GitHub}        & \href{https://github.com/LeeiFrankJaw}{\color{linkcolour}github.com/LeeiFrankJaw}
\end{tabular}

%Section: Work Experience at the top
\section{工作经历}
\begin{tabular}{r|p{11cm}}
  2017年2月    & \href{https://www.mesoor.com/}{麦穗人工智能}(上海)\hfill Python程序员 \\
  至10月       & \emph{爬虫和前端开发}\\
               & \footnotesize 使用pyspider编写爬虫,从主要的中文招聘网站(如\href{http://www.chinahr.com/}{中华英才}、\href{https://www.zhaopin.com/}{智联}、\href{https://www.lagou.com/}{拉勾}等)爬取职位描述。我用W3C WebDriver实现了一个自动登录系统,帮助我们客户从这些平台导入简历到我们的系统,并且向W3C提交了一个\href{https://github.com/w3c/web-platform-tests/pull/6743}{修复}。我参与了前端网站开发,使用Angular/TypeScript和纯JavaScript。我也参与了一些后端开发,使用flask和hug。我也经常被分配一些任务,这些任务与数据库(主要是PostgreSQL,也有一些MySQL)的操作、迁移相关。\\
  \multicolumn{2}{c}{} \\
  2014年11月至 & \href{http://www.starworking.com/}{星畅科技}(上海)\hfill Clojure程序员实习 \\
  2015年2月    & \emph{全栈开发}\\
               & \footnotesize 在一边学习Clojure的过程中,使用Clojure和相关技术完成全站开发。有关我如何得到这次实习,环球时报在当时还有一些\href{http://www.globaltimes.cn/content/871111.shtml}{报道}。
\end{tabular}

%Section: Education
\section{教育}
\begin{tabular}{rp{11.5cm}}
  2017年6月 & 计算机及应用\href{http://www.chsi.com.cn/en/news/201312/20131202/663878204.html}{专科}\quad 上海交通大学\\
            & \footnotesize 这个学历是我通过自学考试得到的。在毕业时,我已经完成了C、RDBMS、数据结构、离散数学、计算机组成、操作系统、网络、数字电路、8086接口技术等课程的学习。\\&\\
  2011年8月至 & \href{http://www.cornellcollege.edu/}{康奈尔学院}\quad 爱荷华州芒特弗农市\\
  2012年2月   & \footnotesize 这个学校和那个康奈尔大学没有任何关系,而是一个独立的文理学院。我在这里只上了一个学期,我学习了Java和形式逻辑并且从此成为了一名Emacs和Linux用户。
\end{tabular}

\section{技能}
\begin{tabular}{rp{12cm}}
  基本 & Bash, GNU工具链, XQuery, x86\,指令集, Redis, MongoDB, Docker, OLAP\\
  中等 & Python, JavasScript, PostgreSQL, MySQL, Clojure, C, Angular, Java, Scala, \LaTeX
\end{tabular}

%Section: Languages
\section{语言}
\begin{tabular}{rl}
  中文 & 母语\\
  英文 & 流利\\
  俄文 & 基本
\end{tabular}

%Section: Awards and additional info
\section{曾获奖项}
\begin{tabular}{rp{11.6cm}}
  2009年11月 & 全国青少年信息学奥林匹克联赛 (NOIP) 二等奖\\
             & 中国计算机学会\\
             & \footnotesize NOIP是一年一度中学生编程竞赛。该竞赛为个人赛,共5个小时,解答3道算法类的问题。
\end{tabular}

\section{慕课经历}
\begin{tabular}{rp{11.6cm}}
  2015年5月  & \href{https://s3.amazonaws.com/verify.edx.org/downloads/5eb2696ec8874930bb0fda0b8b6756b6/Certificate.pdf}{线性代数——从基础到前沿}\hfill\textbf{edX}\\
             & \footnotesize 这是我首次接触MATLAB。我学习了使用 \href{http://www.cs.utexas.edu/~flame/web/}{FLAME} 方法系统地开发稠密线性代数库。\\&\\
  2015年4月  & \href{https://s3.amazonaws.com/accredible_user_certificate/certificates/144339/original/Coursera_matrix_2015.pdf}{编码矩阵:计算机科学应用中的线性代数}\hfill\textbf{Coursera}\\
             & \footnotesize 我学习了线性代数在诸多领域的应用,如计算机视觉、密码学、图形学、信息检索等。\\&\\
  2014年9月  & \href{https://s3.amazonaws.com/accredible_user_certificate/certificates/53611/original/Coursera_hwswinterface_2014.pdf}{软硬件接口}\hfill\textbf{Coursera}\\
             & \footnotesize 我学习了位于高层次语言底下的主要抽象层次、记数系统、汇编语言、内存管理、操作系统进程模型、高层次机器架构——包括存储层次和高层次语言是如何实现的。\\&\\
  2014年6月  & \href{https://www.coursera.org/account/accomplishments/records/jeAXpfyLDdj7TBYK}{Scala函数式编程原理}\hfill\textbf{Coursera}\\
             & \footnotesize 除了Scala的动手实践经验之外,我还学习了函数式程序中不变性的证明和符号地跟踪程序的执行。\\&\\
  2014年6月  & \href{https://www.coursera.org/account/accomplishments/records/AqU3pfW4qRTd8FzE}{逻辑学:语言与信息2}\hfill\textbf{Coursera}\\
             & \footnotesize 这是我的第一个官方认证慕课证书。我学习了形式逻辑在不同领域的各种应用,如EE、CS、语言学、哲学和数学。特别地,我学习了Robinson合一算法和归结,摆弄了SWI-Prolog。这门课也加深了我对\(\epsilon\textit{-}\delta\) 语言的理解,该语言常见于数学分析。\\&\\
  2013年6月  & \href{https://s3.amazonaws.com/verify.edx.org/downloads/eedec1d10b884139876bee106313142c/Certificate.pdf}{计算机科学与编程导论}\hfill\textbf{edX}\\
             & \footnotesize 这门课拓宽了我的视野,向我介绍了计算机科学中的诸多话题。\\&\\
  2012年11月 & \href{https://s3.amazonaws.com/accredible_user_certificate/certificates/48300/original/IntroLogic.pdf}{逻辑学导论}\hfill\textbf{Coursera}\\
             & \footnotesize 我是国内最早的慕课学习者之一。可能是由于我对形式逻辑的终身兴趣,这门课是我最早完成的慕课之一。
\end{tabular}

\end{document}

% Local Variables:
% TeX-engine: xetex
% End:
