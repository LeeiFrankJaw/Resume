\documentclass[a4paper,11pt]{article}

%A Few Useful Packages
\usepackage{marvosym}
\usepackage[math-style=TeX]{unicode-math}
\usepackage{fontspec} 					%for loading fonts
\usepackage{xunicode,xltxtra,url,parskip} 	%other packages for formatting
\usepackage{color,graphicx}
\usepackage[usenames,dvipsnames]{xcolor}
\usepackage[big]{layaureo} 				%better formatting of the A4 page
% an alternative to Layaureo can be ** \usepackage{fullpage} **
\usepackage{supertabular} 				%for Grades
\usepackage{titlesec}					%custom \section
\usepackage{ragged2e}

%Setup hyperref package, and colours for links
\usepackage{hyperref}
\definecolor{linkcolour}{rgb}{0,0.2,0.6}
\hypersetup{
  pdftitle={Lei’s Resume},
  pdfauthor={Lei Zhao},
  pdfkeywords={resume},
  pdfcreator={XeLaTeX},
  hidelinks
}

%FONTS
\defaultfontfeatures{Mapping=tex-text}
%\setmainfont[SmallCapsFont = Fontin SmallCaps]{Fontin}
%%% modified for Karol Kozioł for ShareLaTeX use
\setmainfont{Fontin.otf}[
  SmallCapsFont = Fontin-SmallCaps.otf,
  BoldFont = Fontin-Bold.otf,
  ItalicFont = Fontin-Italic.otf
]
\setmathfont{texgyrepagella-math.otf}
%%%
\usepackage{metalogo}
\setlogokern{La}{-0.18em}
\setlogokern{aT}{-0.09em}
\setlogodrop[TeX]{0.27ex}

\hyphenpenalty=500
\frenchspacing

%CV Sections inspired by:
%http://stefano.italians.nl/archives/26
\titleformat{\section}{\Large\scshape\raggedright}{}{0em}{}[\titlerule]
\titlespacing{\section}{0pt}{3pt}{3pt}
%Tweak a bit the top margin
%\addtolength{\voffset}{-1.3cm}

%Italian hyphenation for the word: ''corporations''
\hyphenation{im-pre-se}

%-------------WATERMARK TEST [**not part of a CV**]---------------
\usepackage[absolute]{textpos}

\setlength{\TPHorizModule}{30mm}
\setlength{\TPVertModule}{\TPHorizModule}
\textblockorigin{2mm}{0.65\paperheight}
\setlength{\parindent}{0pt}

%--------------------BEGIN DOCUMENT----------------------
\begin{document}


\pagestyle{empty} % non-numbered pages


%--------------------TITLE-------------
\par{\centering
		{\Huge Lei \textsc{Zhao}
	}\bigskip\par}

%--------------------SECTIONS-----------------------------------
%Section: Personal Data
\section{Personal Data}

\begin{tabular}{rl}
  % \textsc{Place and Date of Birth:} & Someplace, Italy  | dd Month 1912 \\
  \textsc{Address:}   & Pudong New Area, Shanghai, China \\
  \textsc{Phone:}     & +1 319 555 0128\\
  \textsc{email:}     & \href{mailto:lei.zhao@example.com}{\color{linkcolour}lei.zhao@example.com} \\
  \textsc{StackOverflow:} & \href{https://stackoverflow.com/users/2117531/lei-zhao}{\color{linkcolour}stackoverflow.com/u/2117531} \\
  \textsc{GitHub:} & \href{https://github.com/LeeiFrankJaw}{\color{linkcolour}github.com/LeeiFrankJaw}
\end{tabular}

%Section: Work Experience at the top
\section{Work Experience}
\begin{tabular}{r|p{11cm}}
  \textsc{Feb--Dec 2017} & Python programmer at \href{https://www.mesoor.com/}{\textsc{Mesoor}}, Shanghai \\
                         & \emph{Web Crawler and Front-end Development}\\
                         & \footnotesize I wrote web crawler using pyspider to crawl job descriptions from several major Chinese employment-related website like \href{http://www.chinahr.com/}{chinahr.com}, \href{https://www.zhaopin.com/}{zhaopin.com}, \href{https://www.lagou.com/}{lagou.com}, and so on so forth.  I implemented an automatic login service using w3c \href{https://github.com/w3c/web-platform-tests/pull/6743}{web driver} technology and help our client import their resumes into our own system.  I also participated in a lot front-end website development with both angular/typescript and vanilla javascript and some back-end development with flask and hug.  I was also often assigned tasks related to database (PostgreSQL primarily, also MySQL) operation and migration.\\
  \multicolumn{2}{c}{} \\
  \textsc{Nov 2014--}    & Clojure programmer at \href{http://www.starworking.com/}{\textsc{Starworking}}, Shanghai \\
  \textsc{Feb 2015}      & \emph{Full-stack Development}\\
                         & \footnotesize Develop full-stack website with Clojure and related technology while learning Clojure on the go.  There was also some media coverage by \href{http://www.globaltimes.cn/content/871111.shtml}{Global Times} on how I found this internship.
\end{tabular}

%Section: Education
\section{Education}
\begin{tabular}{rp{11.5cm}}
  \textsc{Sept 2014--} & \href{http://www.chsi.com.cn/en/news/201312/20131202/663878204.html}{Zhuanke Certificate of Graduation} in \textsc{Computer Science}\\
  \textsc{Jun 2017}    & \textbf{SJTU}, Shanghai, China \\
                       & \footnotesize This certificate, which I attained by self-taught examination, is equivalent to an Associate's degree.  Upon completion, I learned C, RDBMS, data structures, discrete math, computer organization, operating system, networking, digital logic, 8086 interfacing, and so on.\\&\\
  \textsc{Aug 2011--}  & \textbf{Cornell College}, Mount Vernon, IA\\
  \textsc{Feb 2012}    & \footnotesize Since I had only one semester here, I learned about Java and formal logic, and became an Emacs and Linux user.
\end{tabular}

\section{Computer Skills}
\begin{tabular}{rp{9cm}}
  Basic Knowledge:& Bash, \textsc{gnu} toolchain, XQuery, x86\,\textsc{isa}, Redis, MongoDB\\
  Intermediate Knowledge:& Python, JavasScript, PostgreSQL, Clojure, C, Angular, Java, Scala, \LaTeX
\end{tabular}

%Section: Languages
\section{Languages}
\begin{tabular}{rl}
  \textsc{Chinese:} & Native\\
  \textsc{English:} & Proficient\\
  \textsc{Russian:} & Rudimentary
\end{tabular}

%Section: Awards and additional info
\section{Honors and Awards}
\begin{tabular}{rp{11.6cm}}
  \textsc{Nov 2009} & 2nd Prize in National Olympiad in Informatics in Provinces (\textsc{noip})\\
                    & \textbf{China Computer Federation}\\
                    & \footnotesize \textsc{noip} is an annual competitive programming competition for secondary school students.  The contest consists of five hours of programming on an individual basis, solving problems of an algorithmic nature.
\end{tabular}

\section{Certifications}
\begin{tabular}{rp{11.6cm}}
  \textsc{May 2015}  & \href{https://s3.amazonaws.com/verify.edx.org/downloads/5eb2696ec8874930bb0fda0b8b6756b6/Certificate.pdf}{Linear Algebra - Foundations to Frontiers}\hfill\textbf{edX}\\
                     & \footnotesize This was my first exposure to MATLAB and I learned about FLAME methodology for systematically developing dense linear algebra library.\\&\\
  \textsc{Apr 2015}  & \RaggedRight \href{https://s3.amazonaws.com/accredible_user_certificate/certificates/144339/original/Coursera_matrix_2015.pdf}{Coding the Matrix: Linear Algebra through Computer Science Applications}\hfill\textbf{Coursera}\\
                     & \footnotesize I learned many applications of linear algebra in a variety of fields like computer vision, cryptography, graphics, information retrieval, and so on.\\&\\
  \textsc{Sept 2014} & \href{https://s3.amazonaws.com/accredible_user_certificate/certificates/53611/original/Coursera_hwswinterface_2014.pdf}{The Hardware/Software Interface}\hfill\textbf{Coursera}\\
                     & \footnotesize I learned key computational abstraction levels below modern high-level languages, number representation, assembly language, memory management, the operating system process model, high-level machine architecture including the memory hierarchy, and how high-level languages are implemented.  I was very sad to know in 2015 that Prof.\ Borriello passed away after six-year fight against colon cancer.\\&\\
  \textsc{Jun 2014}  & \href{https://www.coursera.org/account/accomplishments/records/jeAXpfyLDdj7TBYK}{Functional Programming Principles in Scala}\hfill\textbf{Coursera}\\
                     & \footnotesize In addition to some hands-on experiences with Scala, I learned about proofs of invariants for functional programs and how to trace execution symbolically.\\&\\
  \textsc{Jun 2014}  & \href{https://www.coursera.org/account/accomplishments/records/AqU3pfW4qRTd8FzE}{Logic: Language and Information 2}\hfill\textbf{Coursera}\\
                     & \footnotesize This is my first officially verified MOOC\@.  I learned various application of formal logic in different fields such as EE, CS, linguistics, philosophy, and math.  Specifically, I learned Robinson's unification and resolution and played with SWI-Prolog.  This course also deepened my understanding of \(\epsilon\textit{-}\delta\) language often seen in mathematical analysis.\\&\\
  \textsc{Jun 2013}  & \href{https://s3.amazonaws.com/verify.edx.org/downloads/eedec1d10b884139876bee106313142c/Certificate.pdf}{Introduction to Computer Science and Programming}\hfill\textbf{edX}\\
                     & \footnotesize This course broadened my horizons and introduced me to many topics in computer science.\\&\\
  \textsc{Nov 2012}  & \href{https://s3.amazonaws.com/accredible_user_certificate/certificates/48300/original/IntroLogic.pdf}{Introduction to Logic}\hfill\textbf{Coursera}\\
                     & I was one of earliest MOOC learners in China.  Perhaps due to my life-long interest in formal logic, this course was one of my first MOOCs.
\end{tabular}

\end{document}

% Local Variables:
% TeX-engine: xetex
% End:
